\documentclass[]{extarticle}
\usepackage{lmodern}
\usepackage{amssymb,amsmath}
\usepackage{ifxetex,ifluatex}
\usepackage{fixltx2e} % provides \textsubscript
\ifnum 0\ifxetex 1\fi\ifluatex 1\fi=0 % if pdftex
  \usepackage[T1]{fontenc}
  \usepackage[utf8]{inputenc}
\else % if luatex or xelatex
  \ifxetex
    \usepackage{mathspec}
  \else
    \usepackage{fontspec}
  \fi
  \defaultfontfeatures{Ligatures=TeX,Scale=MatchLowercase}
\fi
% use upquote if available, for straight quotes in verbatim environments
\IfFileExists{upquote.sty}{\usepackage{upquote}}{}
% use microtype if available
\IfFileExists{microtype.sty}{%
\usepackage{microtype}
\UseMicrotypeSet[protrusion]{basicmath} % disable protrusion for tt fonts
}{}
\usepackage[margin=0.6in]{geometry}
\usepackage{hyperref}
\hypersetup{unicode=true,
            pdftitle={Forward and Backward Propagation Example},
            pdfborder={0 0 0},
            breaklinks=true}
\urlstyle{same}  % don't use monospace font for urls
\usepackage{graphicx,grffile}
\makeatletter
\def\maxwidth{\ifdim\Gin@nat@width>\linewidth\linewidth\else\Gin@nat@width\fi}
\def\maxheight{\ifdim\Gin@nat@height>\textheight\textheight\else\Gin@nat@height\fi}
\makeatother
% Scale images if necessary, so that they will not overflow the page
% margins by default, and it is still possible to overwrite the defaults
% using explicit options in \includegraphics[width, height, ...]{}
\setkeys{Gin}{width=\maxwidth,height=\maxheight,keepaspectratio}
\IfFileExists{parskip.sty}{%
\usepackage{parskip}
}{% else
\setlength{\parindent}{0pt}
\setlength{\parskip}{6pt plus 2pt minus 1pt}
}
\setlength{\emergencystretch}{3em}  % prevent overfull lines
\providecommand{\tightlist}{%
  \setlength{\itemsep}{0pt}\setlength{\parskip}{0pt}}
\setcounter{secnumdepth}{0}
% Redefines (sub)paragraphs to behave more like sections
\ifx\paragraph\undefined\else
\let\oldparagraph\paragraph
\renewcommand{\paragraph}[1]{\oldparagraph{#1}\mbox{}}
\fi
\ifx\subparagraph\undefined\else
\let\oldsubparagraph\subparagraph
\renewcommand{\subparagraph}[1]{\oldsubparagraph{#1}\mbox{}}
\fi

%%% Use protect on footnotes to avoid problems with footnotes in titles
\let\rmarkdownfootnote\footnote%
\def\footnote{\protect\rmarkdownfootnote}

%%% Change title format to be more compact
\usepackage{titling}

% Create subtitle command for use in maketitle
\providecommand{\subtitle}[1]{
  \posttitle{
    \begin{center}\large#1\end{center}
    }
}

\setlength{\droptitle}{-2em}

  \title{Forward and Backward Propagation Example}
    \pretitle{\vspace{\droptitle}\centering\huge}
  \posttitle{\par}
  \subtitle{Feb.~10, 2020}
  \author{}
    \preauthor{}\postauthor{}
    \date{}
    \predate{}\postdate{}
  
\usepackage{soul}
\usepackage{booktabs}

\begin{document}
\maketitle

Suppose we are doing a regression problem (so our loss function is mean
squared error) using a network with the structure:

\begin{itemize}
\tightlist
\item
  Input layer has one feature, \(X_1\)
\item
  First hidden layer has two units and a relu activation
\item
  Second hidden layer has two units and a relu activation
\item
  Output layer has one unit and a linear activation
\end{itemize}

For simplicity, suppose I have just one observation with
\(X^{(1)}_1 = 2\) and \(y^{(1)} = 1\).

Also suppose my current estimates of the model parameters are as
follows:

\begin{itemize}
\tightlist
\item
  Layer 1:

  \begin{itemize}
  \tightlist
  \item
    \(b^{[1]}_1 = 0\),
    \(\left(w^{[1]}_1\right)^T = \begin{bmatrix}1\end{bmatrix}\)
  \item
    \(b^{[1]}_2 = 1\),
    \(\left(w^{[1]}_2\right)^T = \begin{bmatrix}-2\end{bmatrix}\)
  \end{itemize}
\item
  Layer 2:

  \begin{itemize}
  \tightlist
  \item
    \(b^{[2]}_1 = 0\),
    \(\left(w^{[2]}_1\right)^T = \begin{bmatrix}0.5 & 10 \end{bmatrix}\)
  \item
    \(b^{[2]}_2 = 1\),
    \(\left(w^{[2]}_2\right)^T = \begin{bmatrix}1 & 1 \end{bmatrix}\)
  \end{itemize}
\item
  Layer 3:

  \begin{itemize}
  \tightlist
  \item
    \(b^{[3]}_1 = 0\),
    \(\left(w^{[3]}_1\right)^T = \begin{bmatrix}1 & 1 \end{bmatrix}\)
  \end{itemize}
\end{itemize}

\hypertarget{draw-a-diagram-of-this-neural-network-model}{%
\paragraph{1. Draw a diagram of this neural network
model}\label{draw-a-diagram-of-this-neural-network-model}}

\vspace{3cm}

\hypertarget{forward-propagation}{%
\paragraph{2. Forward propagation}\label{forward-propagation}}

Find \(a^{(1)[1]}_1\)

\vspace{0.75cm}

Find \(a^{(1)[1]}_2\)

\vspace{0.75cm}

Find \(a^{(1)[2]}_1\)

\vspace{0.75cm}

Find \(a^{(1)[2]}_2\)

\vspace{0.75cm}

Find \(a^{(1)[3]}_1\)

\vspace{0.75cm}

Find the contribution to the MSE from this observation.

\newpage

If I make a small change to the weight vector for the first unit in
layer 2 from
\(\left(w^{[2]}_1\right)^T = \begin{bmatrix}0.5 & 10 \end{bmatrix}\) to
\(\left(w^{[2]}_1\right)^T = \begin{bmatrix}0.5 & 10.1 \end{bmatrix}\),
does this affect the model's final prediction for this observation? What
can you say about
\(\frac{\partial J^{(1)}(b, w)}{\partial w^{[2]}_{12}}\)?

\vspace{3cm}

If I make a small change to the weight vector for the second unit in
layer 1 from
\(\left(w^{[1]}_2\right)^T = \begin{bmatrix}-2\end{bmatrix}\) to
\(\left(w^{[1]}_2\right)^T = \begin{bmatrix}-2.1\end{bmatrix}\), does
this affect the model's final prediction for this observation? What can
you say about \(\frac{\partial J^{(1)}(b, w)}{\partial w^{[1]}_{21}}\)?

\vspace{3cm}

If I make a small change to the weight vector for the first unit in
layer 2 from
\(\left(w^{[2]}_1\right)^T = \begin{bmatrix}0.5 & 10 \end{bmatrix}\) to
\(\left(w^{[2]}_1\right)^T = \begin{bmatrix}0.6 & 10 \end{bmatrix}\),
does this affect the model's final prediction for this observation? What
can you say about
\(\frac{\partial J^{(1)}(b, w)}{\partial w^{[2]}_{11}}\)?

\vspace{3cm}

If I make a small change to the weight vector for the second unit in
layer 2 from
\(\left(w^{[2]}_2\right)^T = \begin{bmatrix}1 & 1 \end{bmatrix}\) to
\(\left(w^{[2]}_2\right)^T = \begin{bmatrix}1.1 & 1\end{bmatrix}\), does
this affect the model's final prediction for this observation? What can
you say about \(\frac{\partial J^{(1)}(b, w)}{\partial w^{[2]}_{21}}\)?

\vspace{3cm}

If I make a small change to the weight vector for the first unit in
layer 1 from
\(\left(w^{[1]}_1\right)^T = \begin{bmatrix}1\end{bmatrix}\) to
\(\left(w^{[1]}_1\right)^T = \begin{bmatrix}1.1\end{bmatrix}\), does
this affect the model's final prediction for this observation? What can
you say about \(\frac{\partial J^{(1)}(b, w)}{\partial w^{[1]}_{11}}\)?


\end{document}
