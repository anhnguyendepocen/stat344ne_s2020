\documentclass[]{extarticle}
\usepackage{lmodern}
\usepackage{amssymb,amsmath}
\usepackage{ifxetex,ifluatex}
\usepackage{fixltx2e} % provides \textsubscript
\ifnum 0\ifxetex 1\fi\ifluatex 1\fi=0 % if pdftex
  \usepackage[T1]{fontenc}
  \usepackage[utf8]{inputenc}
\else % if luatex or xelatex
  \ifxetex
    \usepackage{mathspec}
  \else
    \usepackage{fontspec}
  \fi
  \defaultfontfeatures{Ligatures=TeX,Scale=MatchLowercase}
\fi
% use upquote if available, for straight quotes in verbatim environments
\IfFileExists{upquote.sty}{\usepackage{upquote}}{}
% use microtype if available
\IfFileExists{microtype.sty}{%
\usepackage{microtype}
\UseMicrotypeSet[protrusion]{basicmath} % disable protrusion for tt fonts
}{}
\usepackage[margin=0.6in]{geometry}
\usepackage{hyperref}
\hypersetup{unicode=true,
            pdftitle={matplotlib},
            pdfborder={0 0 0},
            breaklinks=true}
\urlstyle{same}  % don't use monospace font for urls
\usepackage{color}
\usepackage{fancyvrb}
\newcommand{\VerbBar}{|}
\newcommand{\VERB}{\Verb[commandchars=\\\{\}]}
\DefineVerbatimEnvironment{Highlighting}{Verbatim}{commandchars=\\\{\}}
% Add ',fontsize=\small' for more characters per line
\usepackage{framed}
\definecolor{shadecolor}{RGB}{48,48,48}
\newenvironment{Shaded}{\begin{snugshade}}{\end{snugshade}}
\newcommand{\AlertTok}[1]{\textcolor[rgb]{1.00,0.81,0.69}{#1}}
\newcommand{\AnnotationTok}[1]{\textcolor[rgb]{0.50,0.62,0.50}{\textbf{#1}}}
\newcommand{\AttributeTok}[1]{\textcolor[rgb]{0.80,0.80,0.80}{#1}}
\newcommand{\BaseNTok}[1]{\textcolor[rgb]{0.86,0.64,0.64}{#1}}
\newcommand{\BuiltInTok}[1]{\textcolor[rgb]{0.80,0.80,0.80}{#1}}
\newcommand{\CharTok}[1]{\textcolor[rgb]{0.86,0.64,0.64}{#1}}
\newcommand{\CommentTok}[1]{\textcolor[rgb]{0.50,0.62,0.50}{#1}}
\newcommand{\CommentVarTok}[1]{\textcolor[rgb]{0.50,0.62,0.50}{\textbf{#1}}}
\newcommand{\ConstantTok}[1]{\textcolor[rgb]{0.86,0.64,0.64}{\textbf{#1}}}
\newcommand{\ControlFlowTok}[1]{\textcolor[rgb]{0.94,0.87,0.69}{#1}}
\newcommand{\DataTypeTok}[1]{\textcolor[rgb]{0.87,0.87,0.75}{#1}}
\newcommand{\DecValTok}[1]{\textcolor[rgb]{0.86,0.86,0.80}{#1}}
\newcommand{\DocumentationTok}[1]{\textcolor[rgb]{0.50,0.62,0.50}{#1}}
\newcommand{\ErrorTok}[1]{\textcolor[rgb]{0.76,0.75,0.62}{#1}}
\newcommand{\ExtensionTok}[1]{\textcolor[rgb]{0.80,0.80,0.80}{#1}}
\newcommand{\FloatTok}[1]{\textcolor[rgb]{0.75,0.75,0.82}{#1}}
\newcommand{\FunctionTok}[1]{\textcolor[rgb]{0.94,0.94,0.56}{#1}}
\newcommand{\ImportTok}[1]{\textcolor[rgb]{0.80,0.80,0.80}{#1}}
\newcommand{\InformationTok}[1]{\textcolor[rgb]{0.50,0.62,0.50}{\textbf{#1}}}
\newcommand{\KeywordTok}[1]{\textcolor[rgb]{0.94,0.87,0.69}{#1}}
\newcommand{\NormalTok}[1]{\textcolor[rgb]{0.80,0.80,0.80}{#1}}
\newcommand{\OperatorTok}[1]{\textcolor[rgb]{0.94,0.94,0.82}{#1}}
\newcommand{\OtherTok}[1]{\textcolor[rgb]{0.94,0.94,0.56}{#1}}
\newcommand{\PreprocessorTok}[1]{\textcolor[rgb]{1.00,0.81,0.69}{\textbf{#1}}}
\newcommand{\RegionMarkerTok}[1]{\textcolor[rgb]{0.80,0.80,0.80}{#1}}
\newcommand{\SpecialCharTok}[1]{\textcolor[rgb]{0.86,0.64,0.64}{#1}}
\newcommand{\SpecialStringTok}[1]{\textcolor[rgb]{0.80,0.58,0.58}{#1}}
\newcommand{\StringTok}[1]{\textcolor[rgb]{0.80,0.58,0.58}{#1}}
\newcommand{\VariableTok}[1]{\textcolor[rgb]{0.80,0.80,0.80}{#1}}
\newcommand{\VerbatimStringTok}[1]{\textcolor[rgb]{0.80,0.58,0.58}{#1}}
\newcommand{\WarningTok}[1]{\textcolor[rgb]{0.50,0.62,0.50}{\textbf{#1}}}
\usepackage{graphicx,grffile}
\makeatletter
\def\maxwidth{\ifdim\Gin@nat@width>\linewidth\linewidth\else\Gin@nat@width\fi}
\def\maxheight{\ifdim\Gin@nat@height>\textheight\textheight\else\Gin@nat@height\fi}
\makeatother
% Scale images if necessary, so that they will not overflow the page
% margins by default, and it is still possible to overwrite the defaults
% using explicit options in \includegraphics[width, height, ...]{}
\setkeys{Gin}{width=\maxwidth,height=\maxheight,keepaspectratio}
\IfFileExists{parskip.sty}{%
\usepackage{parskip}
}{% else
\setlength{\parindent}{0pt}
\setlength{\parskip}{6pt plus 2pt minus 1pt}
}
\setlength{\emergencystretch}{3em}  % prevent overfull lines
\providecommand{\tightlist}{%
  \setlength{\itemsep}{0pt}\setlength{\parskip}{0pt}}
\setcounter{secnumdepth}{0}
% Redefines (sub)paragraphs to behave more like sections
\ifx\paragraph\undefined\else
\let\oldparagraph\paragraph
\renewcommand{\paragraph}[1]{\oldparagraph{#1}\mbox{}}
\fi
\ifx\subparagraph\undefined\else
\let\oldsubparagraph\subparagraph
\renewcommand{\subparagraph}[1]{\oldsubparagraph{#1}\mbox{}}
\fi

%%% Use protect on footnotes to avoid problems with footnotes in titles
\let\rmarkdownfootnote\footnote%
\def\footnote{\protect\rmarkdownfootnote}

%%% Change title format to be more compact
\usepackage{titling}

% Create subtitle command for use in maketitle
\providecommand{\subtitle}[1]{
  \posttitle{
    \begin{center}\large#1\end{center}
    }
}

\setlength{\droptitle}{-2em}

  \title{matplotlib}
    \pretitle{\vspace{\droptitle}\centering\huge}
  \posttitle{\par}
    \author{}
    \preauthor{}\postauthor{}
    \date{}
    \predate{}\postdate{}
  
\usepackage{soul}
\usepackage{booktabs}

\begin{document}
\maketitle

\hypertarget{background}{%
\paragraph{Background}\label{background}}

matplotlib is one of many Python libraries for plotting. It's not
neessarily my favorite but it is probably the most commonly used.

There are a million ways to do things using matplotlib. We'll use the
pyplot interface; it's relatively easy to use and does everything we
need it to. I'll just show you one way to do make a couple of common
plots.

The examples in this document come from the matplotlib documentation.
When I want to figure out how to make a plot, I find the easiest thing
to do is find an example like what I want to do at
\url{https://matplotlib.org/gallery/}

\hypertarget{import}{%
\paragraph{Import}\label{import}}

It's standard \texttt{import\ matplotlib.pyplot\ as\ plt}. Then you will
refer to functions provided by \texttt{pyplot} using the syntax
\texttt{plt.\_\_\_\_\_}, filling in the blank as necessary.

\begin{Shaded}
\begin{Highlighting}[]
\ImportTok{import}\NormalTok{ numpy }\ImportTok{as}\NormalTok{ np}
\NormalTok{np.random.seed(}\DecValTok{19680801}\NormalTok{)}

\ImportTok{import}\NormalTok{ matplotlib.pyplot }\ImportTok{as}\NormalTok{ plt}
\end{Highlighting}
\end{Shaded}

\hypertarget{line-plots-plt.plot}{%
\paragraph{\texorpdfstring{Line Plots:
\texttt{plt.plot()}}{Line Plots: plt.plot()}}\label{line-plots-plt.plot}}

Line segments connect a series of points

\begin{itemize}
\tightlist
\item
  First argument is a vector of numbers specifying horizontal axis
  coordinates of the points
\item
  Second argument is a vector of numbers specifying vertical axis
  coordinates of the points
\item
  Optional \texttt{color\ =} specifies color
\item
  Optional \texttt{linestyle\ =} specifies linestyle: `-', `--', `-.',
  `:'
\item
  Optional \texttt{linewidth\ =} specifies width of line in pixels
\item
  Optional \texttt{label\ =\ \textquotesingle{}label\textquotesingle{}}
  says what label to use in a legend
\end{itemize}

The easiest way to get a legend to be created is to plot things for
separate groups in separate calls to \texttt{plot}.

\begin{Shaded}
\begin{Highlighting}[]
\NormalTok{x }\OperatorTok{=}\NormalTok{ np.linspace(}\DecValTok{0}\NormalTok{, }\DecValTok{1}\NormalTok{, }\DecValTok{101}\NormalTok{)}
\NormalTok{y1 }\OperatorTok{=}\NormalTok{ x}
\NormalTok{y2 }\OperatorTok{=}\NormalTok{ x}\OperatorTok{**}\DecValTok{2}
\NormalTok{y3 }\OperatorTok{=}\NormalTok{ np.full_like(x, }\FloatTok{0.5}\NormalTok{) }\CommentTok{# vector of same shape as x, filled with all 0.5}

\NormalTok{plt.plot(x, y1, color }\OperatorTok{=} \StringTok{"blue"}\NormalTok{, linestyle }\OperatorTok{=} \StringTok{"-"}\NormalTok{, linewidth }\OperatorTok{=} \FloatTok{0.5}\NormalTok{, label }\OperatorTok{=} \StringTok{"y = x"}\NormalTok{)}
\NormalTok{plt.plot(x, y2, color }\OperatorTok{=} \StringTok{"orange"}\NormalTok{, linestyle }\OperatorTok{=} \StringTok{"--"}\NormalTok{, linewidth }\OperatorTok{=} \DecValTok{3}\NormalTok{, label }\OperatorTok{=} \StringTok{"y = x^2"}\NormalTok{)}
\NormalTok{plt.plot(x, y3, color }\OperatorTok{=} \StringTok{"purple"}\NormalTok{, linestyle }\OperatorTok{=} \StringTok{"-."}\NormalTok{, linewidth }\OperatorTok{=} \DecValTok{5}\NormalTok{, label }\OperatorTok{=} \StringTok{"y = 0.5"}\NormalTok{)}
\NormalTok{plt.legend(loc }\OperatorTok{=} \StringTok{"upper left"}\NormalTok{)}
\NormalTok{plt.show()}
\end{Highlighting}
\end{Shaded}

\includegraphics{matplotlib_files/figure-latex/unnamed-chunk-2-1.pdf}

\hypertarget{scatter-plots-plt.scatter}{%
\paragraph{\texorpdfstring{Scatter Plots:
\texttt{plt.scatter()}}{Scatter Plots: plt.scatter()}}\label{scatter-plots-plt.scatter}}

\begin{itemize}
\tightlist
\item
  First argument is a vector of numbers specifying horizontal axis
  coordinates of the points
\item
  Second argument is a vector of numbers specifying vertical axis
  coordinates of the points
\item
  Optional \texttt{s\ =} specifies point size
\item
  Optional \texttt{color\ =} specifies color
\item
  Optional \texttt{marker\ =} specifies shape used for point
\item
  Optional \texttt{label\ =\ \textquotesingle{}label\textquotesingle{}}
  says what label to use in a legend
\end{itemize}

\begin{Shaded}
\begin{Highlighting}[]
\NormalTok{x }\OperatorTok{=}\NormalTok{ np.linspace(}\DecValTok{0}\NormalTok{, }\DecValTok{1}\NormalTok{, }\DecValTok{10}\NormalTok{)}
\NormalTok{y1 }\OperatorTok{=}\NormalTok{ np.random.random(}\DecValTok{10}\NormalTok{)}
\NormalTok{y2 }\OperatorTok{=}\NormalTok{ np.random.random(}\DecValTok{10}\NormalTok{)}
\NormalTok{y3 }\OperatorTok{=}\NormalTok{ np.random.random(}\DecValTok{10}\NormalTok{)}

\NormalTok{plt.scatter(x, y1, s }\OperatorTok{=} \DecValTok{1}\NormalTok{, color }\OperatorTok{=} \StringTok{"blue"}\NormalTok{, marker }\OperatorTok{=} \StringTok{'.'}\NormalTok{, label }\OperatorTok{=} \StringTok{"y1"}\NormalTok{)}
\NormalTok{plt.scatter(x, y2, s }\OperatorTok{=} \DecValTok{10}\NormalTok{, color }\OperatorTok{=} \StringTok{"orange"}\NormalTok{, marker }\OperatorTok{=} \StringTok{'<'}\NormalTok{, label }\OperatorTok{=} \StringTok{"y2"}\NormalTok{)}
\NormalTok{plt.scatter(x, y3, s }\OperatorTok{=} \DecValTok{20}\NormalTok{, color }\OperatorTok{=} \StringTok{"purple"}\NormalTok{, marker }\OperatorTok{=} \StringTok{'v'}\NormalTok{, label }\OperatorTok{=} \StringTok{"y3"}\NormalTok{)}
\NormalTok{plt.legend(loc }\OperatorTok{=} \StringTok{"upper right"}\NormalTok{)}
\NormalTok{plt.show()}
\end{Highlighting}
\end{Shaded}

\includegraphics{matplotlib_files/figure-latex/unnamed-chunk-3-1.pdf}


\end{document}
